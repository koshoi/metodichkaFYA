\documentclass[14pt]{extreport}
\usepackage[utf8]{inputenc}
\usepackage[russian]{babel}
\usepackage[top=20pt,bottom=70pt,left=40pt,right=40pt]{geometry}
\usepackage{amssymb}
\usepackage{amsmath}
\usepackage{graphicx}
\usepackage[nottoc]{tocbibind}
\renewcommand*\rmdefault{cmr}

\begin{document}
	\begin{titlepage}
		\begin{center}  
			\large{Московский государственный университет имени М.В. Ломоносова\\
			Факультет вычислительной математики и кибернетики}\\
		\end{center}
		\begin{center}
			\vspace{160pt}
			\LARGE{\textbf{Формальные языки и автоматы\\}}
			\LARGE{\textbf{Методичка для сдающих и пересдающих}}
			\vspace{320pt}
		\end{center}
		\begin{center}
			\large{Может, зайдет кому-нибудь}
		\end{center}
	\end{titlepage}
	\newpage
	\tableofcontents
	\newpage
	\chapter*{Предисловие}
		Однажды я сдавал формалки. Не самый приятный опыт в моей жизни. Когда я ждал результатов,
		я сказал, что если не сдам, напишу методичку по этому чудесному предмету.\\
		\\
		Как несложно догадаться, я тогда не сдал.
		\\\\
		По формалкам уже есть отличное руководство в виде решений задач от Тани (не знаю, кто это,
		но если бы ее не было, статистика сдаваемости была бы гораздо хуже, я уверен, Таня,
		спасибо, что ты есть), и казалось
		бы - зачем я это делаю?\\
		Ну во-первых, я сказал, что напишу.\\
		Во-вторых, это довольно знатный способ подготовки к пересдаче, который потенциально
		поможет каким-нибудь людям после меня.\\
		В-третьих, наверное, есть смысл восполнить пробел в нормальном "печатном" пособии по
		формальным языкам. Ахо Ульмана я не читал (а он, может быть, и норм), но\ "Теорию
		Построения Комплияторов"\ читать совершенно невозможно.
		\\\\
		В этой методичке я в меру своих возможностей постараюсь не только подробно
		расписать решения различных задач (это уже есть у Тани), но и более-менее человеческим
		языком расписать, как применяемые алгоритмы работают (потому что выучить алгоритм, если
		есть примерное понимание работы, гораздо проще)
		\\\\
		За кривой русский язык извиняйте - я ЕГЭ сдал на 30 баллов.
	\newpage
	\chapter*{Немного теормина}
	Недетерменированный конечный автомат - НКА
	\newpage
	\chapter*{НКА по регулярному выражению}
	\section*{Алгоритм}
	Построение НКА по РВ делается по определенным правилам\\
	Каждое РВ можно разбить на подвыражения, а для простейших подвыражений автомат строится
	довольно просто\\
	На рисунке далее\\
	1. Е-переход\\
	2. Переход по символу $a$\\
	3. Переход по $a|b$ ($a$ или $b$)\\
	4. Переход по $ab$ (сначала $a$, потом $b$)\\
	5. Переход по $a*$ (сколько угодно символов $a$, в т.ч. 0)\\
	Справа от простейших НКА нарисованы общие НКА, когда переход совершается
	по некоторому РВ, где $r$ - РВ, а $M(r)$ - НКА, построенный по нему.\\
	\large{РИСУНОК1}\\
	На рисунке изображено построение НКА для РВ $b(a|ba)*|aab$\\
	\large{РИСУНОК2}\\
	\section*{Почему это работает}
	Правильность таких построений довольно очевидна, но если не верится, можно походить по
	полученным НКА по дугам и убедиться, что ничего кроме того, что описано в РВ, по ним
	составить нельзя.
\end{document}
